\section{Survey: Ontology-based Data Integration}
\label{sec:Survey}
\vspace{-.3cm}
5 Discussion and Conclusions
This section discusses the advantages and disadvantages of ontology-based DI approaches and provides concluding remarks.
\subsection{Discussion of Strengths and Weakness of Ontology-based DI
Approaches} Section 4 provided discussion of uses cases of ontology-based approaches for DI in industry and literature. In this sub-section we are going to discuss advantages and disadvantages of ontology-based DI approaches. Ontology-based DI approaches using ontologies have some benefits that are as follows
It is evident from the industrial use cases cited in Section 4 that the ontologies employed as the global schema are either an existing domain ontology or foundational ontology. The global schema provides enough coverage in the conceptualization of the domain to cater for heterogeneity from the potential data sources. The vocabulary provided by the global ontology serves as a stable conceptual interface to the databases and independent of the database schemas for DI
Ontology-based DI approaches apply semantic meaning and expressiveness to the data. It is possible to perform semantic integration of the heterogeneous data. Ontology-based DI approaches also enable consistent management and recognition of inconsistent data.
Ontology-based DI approaches advocate the utilization of reusable domain concepts. The language by the ontology is expressive enough to address the complexity of queries typical of decision support applications.
Unlike approaches utilizing databases for DI, ontology-based DI approaches take advantage of ontologies that are designed with the purpose of improving interoperability. Based on completeness of a centralized or global schema/ontology (i.e., in the case of hybrid approach) it is possible to achieve data integrity (i.e., data completeness, accuracy and consistency). This leads towards intended DI. Security of accessing the data resources can be achieved by applying local security policies (e.g., local authentication and certificate exchanges).
It is also possible to take advantage of proven manufacturing and production standards (e.g., ISO-10303 STEP, STEP-AP242), OPC-DA for data access and OPC-UA for Machine-to-Machine (M2M) communication protocol.
Although, ontology-based DI approaches provide an efficient DI solution using common vocabulary for domain concepts. They still lack the following,
To design and implement an ontology-based DI approach one has to have an adequate knowledge of semantics expressed in the ontology. Large and complex ontologies are more difficult to understand leading towards extra effort.
Large and complex ontologies can be restrictive to be accepted by new parties involved for DI by asserting strong ontological commitments through complicated and tightly-coupled axiomatization. In such a situation ontologies might not be accepted by potential parties that joined later for DI purposes. This also leads to future extendibility issues.
In the use cases where there is a no conceptual model/ontology available for the application. One has to develop a new ontology from scratch and this requires a large and complicated manual effort especially when one has to capture concepts and relationships between them in a large domain. This situation also contributes towards a large manual effort to design and develop DI approach.
One of the problems with ontology-based approaches arises when there is not much of semantic understanding of the data sources required by the user. An example scenario can be when the user wants to perform manual analysis of the data integrated from various data sources (i.e., weather sensors) by putting it in simple csv files or excel sheets. In such a scenario, there is not much need of having semantics involved. In this situation the ontology-based approach might not be a viable option to take.
It is a challenge to manage interfacing via explicit rules or writing adapters between local ontology and global ontology for DI. The interfacing should be consistent and vocabulary to be updated continuously in both the local and global ontology along with the mapping rules.
Global ontology or schema grows as new enterprise ontologies are added representing heterogeneous data resources (locally from shop-floor or from geographically distributed enterprises). This also requires new mappings to be defined that can introduce more manual effort to develop and maintain new mappings.
Data sharing is only possible upon the availability of global schema/ontology (in the case of hybrid approach) and also the mappings between local and global ontologies. Hence DI depends on availability of the global schema.
Translation of real time sensory data to ontology requires additional cleansing of temporal data. Validity of global schema depends on the correctness of the local ontology
\subsection{Concluding Remarks} 
Smart manufacturing vicinity is a data-driven operational environment. Data-driven operational environment requires data produced by various sources in heterogeneous formats to be integrated, understood and analysed in order to provide operational intelligence. In the context of DI, we have provided a survey focusing on ontology-based DI approaches that can support smart manufacturing by providing semantic integration. We summaries the identified literature provided comparison and discussed strengths and weakness of them. Ontology-based approaches are also well suited when, for instance, there is a requirement of semantic integration among heterogeneous data sources, semantic understanding of integrated data for data analytics and standardization of communication using proven standards like that of STEP, OPC is also a requirement. 
However, there are industrial use cases where ontology-based approaches are might be not feasible, for instance, 1) when there doesn’t exist a conceptual model upfront of the domain leading towards additional effort of creating ontology from scratch, 2) when stakeholder doesn’t require any semantic information of the integrated data and wants to perform manual analysis and 3) when data sources from heterogeneous manufacturing sites finds the global ontology to be very restrictive and rigid to map local domain concepts residing in local ontology. 
The takeaway message from this survey is that although ontology-based DI approaches proved an effective way for semantic integration in manufacturing domain. Ontologies can only be utilized based on context and availability of certain knowledge upfront and not in all industrial use cases. An example industrial use case could be when there is a requirement of manually performing data analytics by performing semantic DI that resides in plain CSV files. In this scenario, semantic integration will prove to be annotation or semantic information intensive. 
In order to improve the scalability and data availability issues, one can utilize the cloud computing paradigm and provide DI as a service (e.g., work by Bohlouli et al. [17]).  
For future work, we would like to add more DI approaches for semantic integration of data sources in manufacturing domain, provide empirical evidence to support our claim and apply systematic literature review to enhance our survey.