\section{Literature Review Methodology}
\label{sec:litRevMethod}
\vspace{-.3cm}
The aim of this Section is to familiarize the reader with the literature collection process and formulation of research questions.
\subsection{Research Process}
This sub-section discusses the investigation and analysis method inspired by the
Review and Analysis method is utilized for literature collection in the
context of smart manufacturing formulated and suggested by Kang.et.al.[33].
However, we have tailored that to our needs in order to provide a formal way of searching and analysing the DI techniques/approaches for smart manufacturing. 
The overall procedures are shown in Figure 1. 
The process starts by 1) searching and reviewing the DI approaches in SM domain, 2) selecting from the identified DI approaches targeting semantic DI, 3) categorizing the identified techniques under the main headings (e.g. approaches providing 1) reference model, 2) frameworks or 3) languages) and 4) discussion of the categorized and classified approaches. 
Discussion and analysis is performed by answering the research questions formalized in Section 2.2. 
We have not utilized a systematic literature review rather we have searched for ontology-based DI approaches and technologies. 
This limitation of selecting the research work has let us focus on only on the semantic DI approaches.
\subsection{Research Focus}
This sub-section provides a discussion on the research question formulated in order to initiate and facilitate the literature search and collection process. The main research question is formulated with the narrow scope which is intentional. The main reason for having the research question with a narrow scope is to serve the purpose of identification of literature on ontology-based DI approaches. The research question not only helped us to narrow the scope but also helped us to classify the identified approaches.
Main Research Question: “How are ontology-based data integration (DI) approaches utilized to support smart manufacturing?”
To answer the main research question, we have formulated two supportive/auxiliary questions (SQs). These supportive questions relates to various aspects of ontology-based DI approaches in smart manufacturing. The formulated smaller questions also address and answer different aspects of the main research question.
SQ1: What ontology-based DI approaches/techniques suggested by the literature to support smart manufacturing?
Rationale: The main idea of formulating this research question is to initiate exploring the current literature on data integration and focus on approaches that are focusing on ontology-based DI approaches that can support SM.
SQ2: What are the weakness and strengths of the Identified ontology-based DI approaches/techniques?
Rationale: The main idea of this supportive question to identify the strengths and weakness of ontology-based approaches and identify the use cases where semantic integration may or may not be applicable.