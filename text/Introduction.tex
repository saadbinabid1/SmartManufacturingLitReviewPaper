\section{Introduction}
\label{sec:intro}
\vspace{-.3cm}
“Smart manufacturing (SM) is the application of advanced communication systems to conventional manufacturing processes, making them more flexible, efficient and responsive”1. Smart manufacturing vicinity is a data-driven operational environment. Data-driven operational environment requires data produced by various sources in heterogeneous formats to be integrated, understood and analysed in order to provide operational intelligence. This intelligence can reduce the resource (e.g., energy, natural resources) usage, increase efficiency, productivity and minimize the time-to-market [32]. Various authors have made attempts to facilitate smart manufacturing, for instance O’Donovan et al. [30-31] focused on equipment maintenance and provide industrial big data pipeline architecture. Data analytics applications based on the designed big data pipeline is suggested by the authors to achieve efficient maintenance of manufacturing equipment.
Data sources can be heterogeneous in syntax, schema, or semantics, thus making data integration and interoperability a difficult task [29]. Data heterogeneity can be classified as 1) Syntactic heterogeneity (i.e., use of different models or languages), 2) Schematic heterogeneity (i.e., structural differences) and 3) Semantic heterogeneity (i.e., different meanings or interpretations of data in various contexts) [29]. Ontology research is another discipline that deals with semantic heterogeneity in structured data and provides a conceptual representation of the data and of their relationships to eliminate possible heterogeneities [29]. The data sources to be integrated can be classified as 1) a real-time data sources (e.g., sensors) or 2) a pre-existing databases (i.e., legacy databases or log files).
Without the Semantic DI, it is difficult to get the context and meaning of the data altogether. It can also become a difficulty for the stakeholders (e.g., operational manager, maintenance engineer, quality control experts) in the manufacturing environment to 1) perform data analytics in order to analyse, 2) comprehend the data and perform efficient decisions to increase time-to-market and 3) reduce the resource consumption (e.g., electricity from grid) to lower the equipment maintenance costs. In this context, we have performed a survey on ontology-based approaches that can support smart manufacturing by enabling semantic DI of the data produced by heterogeneous data sources (e.g., Sensors or databases). We have also discussed the strengths and weakness of the Ontology-based approaches and discussed the use cases where ontology-based approaches might not be feasible. 
The reminder of the paper is as follows, Section 2 discusses our literature collection methodology and research questions, Section 3 provides a brief overview on Ontologies, in Section 4 we summarise the identified approaches and provide comparison between them,  Section 5  provides discussion of strengths and weakness of the ontology-based approaches and concludes the paper.
