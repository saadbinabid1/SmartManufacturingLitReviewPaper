\section{Survey: Ontology-based Data Integration}
\label{sec:Survey}
\vspace{-.3cm}
This Section discusses the industrial use cases in manufacturing domain where ontology-based approaches are applied for DI and provides a comparison between them. A few of the comparison criteria factors are associated with guidelines provided by smart manufacturing leadership coalition (SMLC1). These comparison factors are, 1) Scalability, 2) Interoperability and 3) security. Rest of the comparison factors are perceived while reading the papers (i.e., 1) type of ontology utilized, 2) data source addition,3) domain in which utilized, 4) temporal data processing, 5) evaluation, 6) capacity to manage large data 7) types of data sources for DI and 8) data integrity loss). Our data collection process is inspired by Kang et al. [33]. Kang et al. [33] developed this framework to identify literature on smart manufacturing. We tailored work by Kang et al. [33] for our purpose to identify the literature on semantic integration of data sources in manufacturing domain that can support smart manufacturing paradigm (Section 2.1).
Ontologies have been extensively used in DI systems because they provide an explicit and machine-understandable conceptualization of a domain [29]. This survey is considering ontology-based DI approaches that can support smart manufacturing. The following paragraphs will provide a brief overview of the approaches (frameworks, architecture and process models) that has taken ontologies as a primary way to integrate product information among data sources in enterprises.
Work by Jian et al. [1], Chang and Terpenny[6],Bohlouli et al. [17], HEFKE et al. [3], Uzdanaviciute and Butleris [8], Fang and Wang [12] and Gangon [39] provide hybrid ontology-based DI approach, whereas,  Alm et al. [38] provides a single ontology-based approach for DI. All of the approaches implementing hybrid ontology-based approach DI provide scalability because of the fact that it is relatively easier to add new data sources by adding a local ontology which represents the local/enterprise data source to be integrated. However, this gets manually effort demanding when it comes to single ontology (e.g., Alm et al. [38]). Most of the identified approaches attempt to solve the DI challenge in the manufacturing domain. Most of the approaches (e.g., Jian et al. [1], Chang and Terpenny[6],Bohlouli et al. [17], HEFKE et al. [3], Uzdanaviciute and Butleris [8], Fang and Wang [12] and Gangon [39]) provide DI between data sources that is product-related information residing in databases. The data sources can also be simple digital annotations and operational procedures (i.e., in Alm et al. [38]). 
Almost none of the identified approaches attempted to perform temporal data processing (i.e., temporal data is the data produced by sensors). Most of the identified approaches achieved interoperability by providing mappings between local and global ontologies. Only a few approaches have taken security into consideration and provided mechanism to deal with security related issues. For instance, Jiang et al. [1] provided security through access control layer of the proposed framework. Whereas Bohlouli et al. [17] provide cloud security layer and suggested a security mechanism (e.g., public key encryption, secure sockets layer (SSL) in their proposed cloud-based layered framework. Capacity to manage large data is also an important factor because as discussed earlier smart manufacturing is a data-driven environment. None of the approaches addressed the capacity to manage large data factor. Only work by Bohloui et al. [17] addressed this factor by providing a cloud based approach and implementing DI as a Service. 
Most of the identified approaches provide prototypes as a proof-of-concepts. Apart of a few approaches that utilize either manufacturing enterprises (e.g., Fang and Wang [12] and Chang and Terpenny [6]) or commercial tools (e.g., Plant@Hand in Alm et al. [38]). Almost all identified approaches for DI enables minimum data integrity loss because of the fact that there is a global schema and local ontologies present to define the concepts of the domain and provide semantics. Also, interoperability is achieved through mappings between local and global ontologies. Table 1 provides a comparison between the identified ontology-based DI approaches. 
Other noticeable works are by Noy [5], Wache [10], Cruz and Xiao [29] and Mansukhlal and Malathy [11] that provide surveys on semantic integration using ontologies.